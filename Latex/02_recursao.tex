\documentclass[a4paper, twocolumn]{article}

\usepackage[utf8]{inputenc}
\usepackage{mathtools}
\usepackage{amsmath}
\usepackage{amsthm}
\usepackage{color}
\usepackage{a4wide}
\usepackage{float}
\usepackage{graphicx}
\usepackage{indentfirst}
\usepackage{enumerate}
\usepackage{url}
\usepackage{listings}
\usepackage{caption}

\lstset{
	language=Python,
	basicstyle=\small\sffamily,
	numbers=left,
	numberstyle=\tiny,
	xleftmargin = 3em,
	frame=tb,
	columns=fullflexible,
	showstringspaces=false,
}

\graphicspath{ {images/} }
\theoremstyle{definition}
\newtheorem{definition}{Definition}[section]

\title{Recursão}
\author{Vinicius A. Matias}
\date{\today}
\begin{document}
	\maketitle
	

\section{Introdução}
	Recursão é uma poderosa técnica de programação que tipicamente torna os códigos mais legíveis, ainda que com algumas consequências (principalmente no consumo de memória). Tal técnica utiliza os conceitos de indução matemática e podem ter sua complexidade estudada pelas funções de recorrência.
	
\section{Indução matemática}
	A indução é uma técnica matemática para provar um teorema $T$ para todos os valores de $n$ a partir de um $n_0$ base. Para provar um teorema por indução devem ser obedecidas duas condições:

\begin{enumerate}
	\item Passo base: $T$ é válido para um $n$ mínimo;
	\item Passo indutivo: Para todo $n$ a partir do $n$ base, se $T$ é válido para $n-1$, então $T$ é válido para $n$.
\end{enumerate}

\subsection{Exemplo 1 }
	Considerando a soma dos primeiros $n$ números naturais como $S(n) = 1 + 2 +...+ n$; queremos provar por indução que:
	
	$S(n) = \frac{n*(n+1)}{2}, \forall n \geq 1$ \\
	
	\textbf{Passo base:} $S(1) = 1$
	
	$S(1) = \frac{1*(1+1)}{2} = 1$ \\
	
	\textbf{Passo Indutivo:}
	
	Como o caso base é verdadeiro, assumimos:
	
	$S(n-1) =\frac{(n-1) * ((n-1)+1)}{2}$ verdadeiro por Hipótese de Indução 
	
	Então, para encontrarmos $S(n)$ conhecendo $S(n-1)$ devemos notar que precisamos adicionar $n$ à $S(n-1)$, ou seja:
	
	$S(n) = S(n-1) + n$
	
	E isso é verdade pois:
	
	$S(n) = S(n-1) + n$
	
	$S(n) = \frac{(n-1) * ((n-1)+1)}{2} + n$
	
	$S(n) = \frac{(n-1) * n}{2} + n$
	
	$S(n) = \frac{n^2-n}{2} + n$
	
	$S(n) = \frac{n^2-n + 2n}{2}$
	
	$S(n) = \frac{n^2 + n}{2}$
	
	$S(n) = \frac{n(n + 1)}{2}$
	
	Assim, está demonstrado que $S(n) = \frac{n(n + 1)}{2}$ é a fórmula para a soma dos primeiros $n$ números naturais.

\subsection{Exemplo 2}
	Provar por indução que:
	
	$2^n = 2^{n-1} + 2^{n-2} + ... + 2^1 + 2^0 + 1$ \\
	
	\textbf{Passo base}: n = 1
	
	$2^1 = 2^{1-1} + 1$
	
	$2^1 = 2$ \\
	
	\textbf{Passo indutivo}:
	
	$ST(n-1) = 2^{n-1}$ (verdadeiro por Hipótese de Indução)
	
	Portanto, precisamos notar que dado $ST(n-1)$ podemos definir $ST(n)$ como:
	
	$ST(n) = S(n-1) + 2^{n-1}$
	
	$ST(n) = 2^{n-1} + 2^{n-1}$
	
	$ST(n) = 2*2^{n-1}$
	
	$ST(n) = 2^{n-1+1}$
	
	$ST(n) = 2^n$
	
	Assim, provamos pelo passo indutivo que se $ST(n-1) = 2^{n-1}$, $ST(n) = 2^n$.

\section{Recursão}
A recursão se baseia na indução matemática. Quando sabemos resolver um caso simples do problema, podemos fazer chamadas sequenciais do problema (computacionalmente, uma função) até que consigamos chegar em um ponto simples e de solução direta.

Implementações recursivas ocupam mais memória que um método iterativo, pois as chamadas aos métodos são empilhadas até se chegar ao caso base.

\subsection{Exemplo}
Problema: Usando apenas o operador de adição, definir a multiplicação de dois números inteiros não negativos $m$ e $n$.

\textbf{Caso Base}: Se $m=1$ e $n=0$, temos $m*n = 1*0 = 0$

Mult(0) = 0

Mult(n-1) é verdadeiro por hipótese de indução.

Perceba que a multiplicação também pode ser expressa como:

$m*n = m*(n-1) + ???$

$m*n = m*(n-1) + m$

Assim, podemos definir um código como o seguinte:

\begin{lstlisting}[label=max_array,caption= Multiplicação recursiva]
def mult(m, n):
	if n == 0: return 0
	return mult(m, n-1) + m
\end{lstlisting}

Apenas para comparação, um código iterativo para a multiplicação é exibido na Listing 2.
\begin{lstlisting}[label=max_array,caption= Multiplicação iterativa]
def mult(m, n):
	soma = 0	
	for i in range(n):
		soma += m
	return soma
\end{lstlisting}



\end{document}