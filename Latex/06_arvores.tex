\documentclass[a4paper, twocolumn]{article}

\usepackage[utf8]{inputenc}
\usepackage{mathtools}
\usepackage{amsmath}
\usepackage{amsthm}
\usepackage{color}
\usepackage{a4wide}
\usepackage{float}
\usepackage{verbatim}
\usepackage{graphicx}
\usepackage{indentfirst}
\usepackage{enumerate}
\usepackage{url}
\usepackage{listings}
\usepackage{caption}

\lstset{
	language=Python,
	basicstyle=\small\sffamily,
	numbers=left,
	numberstyle=\tiny,
	xleftmargin = 3em,
	frame=tb,
	columns=fullflexible,
	showstringspaces=false,
}

\graphicspath{ {images/} }
\theoremstyle{definition}
\newtheorem{definition}{Definition}[section]

\title{Árvores}
\author{Vinicius A. Matias}
\date{\today}
\begin{document}
	\maketitle
	
	\section{Introdução}
	Árvores são estruturas de dados que na maioria dos casos visam aumentar a eficiência na busca de elementos armazenados. Elas podem seguir diferentes abordagens, e aqui passaremos por algumas das possíveis. 
	
	\section{Conceitos básicos}
	Árvores tem elementos centrais que dividem elementos entre esquerda e direita (normalmente). O nó que faz essa primeira divisão é chamado de raíz. Abaixo da raíz temos subárvores, e no final de cada subárvore temos os nós folha. Para cada nó da árvorem o número de subárvores geradas é chamado de grau do nó.
	
	A altura de uma árvore é o maior comprimento entre a raíz e um nó folha. Também pode ser incluído para a altura de nós, seguindo o mesmo princípio (comprimento para o nó folha).  A profundidade é o processo inverso, ou seja, a distância de um nó até a raíz. 
	
	Árvores binárias são as que tem apenas um ou dois descendentes. Se o grau é a quantidade de descendentes de um nó, árvores binárias precisam ter nós de, no máximo, grau 2.

\section{Árvore binária de busca}
Uma árvore de binária que visa a busca eficiente terá, assim como nas outras árvores, sua implementação baseada na passagem da raíz. A raíz é um nó, e cada nó de uma árvore deverá ter acesso à seus descendentes à esquerda e à direita.

\begin{lstlisting}[label=abp_no,caption= Nó de uma árvore binária de pesquisa]
class No:
    def __init__(self, ch, esq, dir):
        self.chave = ch
        self.esq = esq
        self.dir = dir
\end{lstlisting}




	
\end{document}