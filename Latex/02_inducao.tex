\documentclass[a4paper, twocolumn]{article}

\usepackage[utf8]{inputenc}
\usepackage{mathtools}
\usepackage{amsmath}
\usepackage{amsthm}
\usepackage{color}
\usepackage{a4wide}
\usepackage{float}
\usepackage{graphicx}
\usepackage{indentfirst}
\usepackage{enumerate}
\usepackage{url}
\usepackage{listings}
\usepackage{caption}

\lstset{
	language=Python,
	basicstyle=\small\sffamily,
	numbers=left,
	numberstyle=\tiny,
	xleftmargin = 3em,
	frame=tb,
	columns=fullflexible,
	showstringspaces=false,
}

\graphicspath{ {images/} }
\theoremstyle{definition}
\newtheorem{definition}{Definition}[section]

\title{Indução}
\author{Vinicius A. Matias}
\date{\today}
\begin{document}
	\maketitle
	
\section{Indução matemática}
A indução é uma técnica matemática para provar um teorema $T$ para todos os valores de $n$. Para provar um teorema por indução devem ser obedecidas duas condições:

\begin{enumerate}
	\item Passo base: $T$ é válido para $n$ = 1
	\item Passo indutivo: Para todo $n$>1, se $T$ é válido para $n-1$, então $T$ é válido para n
\end{enumerate}

\subsection{Exemplo}
Considerando a soma dos primeiros $n$ números naturais como $S(n) = 1 + 2 +...+ n$; queremos provar por indução que:

$S(n) = \frac{n*(n+1)}{2}, \forall n \geq 1$ \\

\textbf{Passo base:} $S(1) = 1$

$S(1) = \frac{1*(1+1)}{2} = 1$ \\

\textbf{Passo Indutivo:}

Como o caso base é verdadeiro, assumimos:

$S(n-1) =\frac{(n-1) * ((n-1)+1)}{2}$ verdadeiro por Hipótese de Indução 

Então, para encontrarmos $S(n)$ conhecendo $S(n-1)$ devemos notar que precisamos adicionar $n$ à $S(n-1)$, ou seja:

$S(n) = S(n-1) + n$

E isso é verdade pois:

$S(n) = S(n-1) + n$

$S(n) = \frac{(n-1) * ((n-1)+1)}{2} + n$

$S(n) = \frac{(n-1) * n}{2} + n$

$S(n) = \frac{n^2-n}{2} + n$

$S(n) = \frac{n^2-n + 2n}{2}$

$S(n) = \frac{n^2 + n}{2}$

$S(n) = \frac{n(n + 1)}{2}$

Assim, está demonstrado que $S(n) = \frac{n(n + 1)}{2}$ é a fórmula para a soma dos primeiros $n$ números naturais.
	
\end{document}